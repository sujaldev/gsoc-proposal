% ------------------------------------------------------------------------------------------------------------------- %

My goal with this project is to improve both user experience and developer experience with mitmproxy.\ The
following is a complete breakdown of my goals with this project along with it's schedule, I have also kept a little
scope for other ideas that may popup during the project.

% ------------------------------------------------------------------------------------------------------------------- %

\subsection{Week 1 -- \texttt{browser.start} command}
\label{subsec:week-1}

I will start the project with a relatively easier issue \href{https://github.com/mitmproxy/mitmproxy/issues/5247}{\#
5247},
this will implement support for more applications such as the firebox web browser to be launched with mitmproxy
pre-configured with the \texttt{browser.start} command.

% ------------------------------------------------------------------------------------------------------------------- %

\subsection{Week 2 -- RPM Packaging}
\label{subsec:week-2}

Implement automated packaging for \texttt{.rpm} based distributions, creating both a stable release and a nightly
branch with GitHub actions.\ For Red Hat family distributions such as Fedora and CentOS, the quickest way to setup
distribution would be via Fedora Copr.\ While I'm unfamiliar with adding packages to default fedora repositories,
I still want to try getting mitmproxy available as a default pacakge.\ I am doing this as one of the first things
because then over the course of my project I'll be able to spot quirks/bugs with packaging for various
distributions and fine tune the automated packaging pipeline.

% ------------------------------------------------------------------------------------------------------------------- %

\subsection{Week 3 -- DEB Packaging}
\label{subsec:week-3}

Add automated packaging for \texttt{.deb} based distributions.\ If I finish the \texttt{.rpm} and \texttt{.deb}
packaging early then this week I want to work on packaging for either other linux distributions or entirely
switch focus from linux to freeBSD.\ Having actively maintained packaging for freeBSD would facilitate an idea I
would like to experiment with, integration of mitmproxy with
pfSense\footnote{\href{https://www.pfsense.org}{pfsense.org}}.\ This might be a simple and easy way to intercept
network wide traffic, simply install mitmproxy package on your pfSense router and a certificate on the client.\
With pfSense, the proxy could also be advertised via DHCP.\

% ------------------------------------------------------------------------------------------------------------------- %

\subsection{Week 4 -- Pre-commit hooks}
\label{subsec:week-4}

Pre-commit hooks to run various tests instead of GitHub actions could save a lot of time and improve developer
experience for the good.

% ------------------------------------------------------------------------------------------------------------------- %

\subsection{Week 5 -- Hex Editor}
\label{subsec:week-5}

Implement a hex editor for binary content.\ \href{https://github.com/mitmproxy/mitmproxy/issues/5231}{\#5231}

% ------------------------------------------------------------------------------------------------------------------- %

\subsection{Week 6 -- Statistics Addon}
\label{subsec:week-6}

Implement a statistics addon for both mitmproxy and mitmweb, this addon will visualize statistics like how many
requests to a particular domain or sub-domain in bar charts, pie charts, etc.\

% ------------------------------------------------------------------------------------------------------------------- %

\subsection{Week 7 -- HAR Import/Export}
\label{subsec:week-7}

Implement import/export for HAR files.\ \href{https://github.com/mitmproxy/mitmproxy/issues/1477}{\#1477}

% ------------------------------------------------------------------------------------------------------------------- %

\subsection{Week 8 -- Browser Integration}
\label{subsec:week-8}

Implement selenium based integration with chrome and firefox to allow for a group/tree view in both mitmproxy and
mitmweb based on which browser tab initiated the request, improving user experience.\ Integration like this could
also unlock other potential features.

% ------------------------------------------------------------------------------------------------------------------- %

\subsection{Week 9 -- Fix Old Bugs}
\label{subsec:week-9}

\begin{itemize}

    \item Fix \href{https://github.com/mitmproxy/mitmproxy/issues/5369}{\#5369}, forward client offer during TLS
    ALPN negotiation in reverse proxy mode.\

    \item Fix \href{https://github.com/mitmproxy/mitmproxy/issues/2989}{\#2989}, trailing-dot domains don't work.

\end{itemize}

% ------------------------------------------------------------------------------------------------------------------- %

\subsection{Week 10--11}
\label{subsec:week-10-11}

These two weeks I want to tackle websocket functionality in mitmproxy such as replay and intercepting.\ I'm not
as familiar with websockets so I might have to request extending the project deadline.

% ------------------------------------------------------------------------------------------------------------------- %

\subsection{Week 12 -- Wrapping Up/Documentation}
\label{subsec:week-12}

Ending the project by writing documentation about new features I've implemented, will also try to implement any
ideas that might have come up during the project this week.\

% ------------------------------------------------------------------------------------------------------------------- %